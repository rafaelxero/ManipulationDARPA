\section{Conclusions}
	\label{sec:conclusions}
	
	The task-level teleoperation strategy presented in this paper was successfully applied
	to the HRP-2Kai humanoid robot, which was able to carry out all the manipulation tasks
	proposed for the DRC, even though there was no accurate model of the environment.
	Our principal restriction was the time limit, considering that most of the time spent
	during the task was the one required by the manual identification of the objects in the
	environment and by the verification of the robot motion by the operator.

	With respect to the door task, it was required to manually align the door lever to the
	point cloud, as well as the grasp position.
	Those operations must be automated to make the robot able to pass through the door faster.
	For the plug task, it is worth to notice that even though its execution lasted 16:34 minutes,
	the effective time was just 1:34 and the number of adjustments required to insert the plug was
	only 3.
	This was the result of assuring a stable grasp of the plug.
	
	As a future work we want to improve and automate our recognition system,
	in order to speed up the execution of every task, as well as to introduce compliant control
	to increase robustness and reliability.