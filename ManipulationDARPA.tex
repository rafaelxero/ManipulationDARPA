\documentclass[letterpaper, 10 pt, conference]{ieeeconf}

\IEEEoverridecommandlockouts
\overrideIEEEmargins

\usepackage[tbtags]{amsmath}
\usepackage{amsfonts}

\renewcommand{\vec}[1]{\boldsymbol{#1}}
\newcommand*{\R}[1]{\mathbb{R}^{#1}}

\def\figurename{Fig.}

\begin{document}

	\title{\LARGE \bf Marker Based Teleoperated Manipulation}

	\author{
		\authorblockN{Rafael Cisneros$^{1}$}
		\authorblockA{rafael.cisneros@aist.go.jp}
		\and
		\authorblockN{Takeshi Sakaguchi$^{1}$}
		\authorblockA{sakaguchi.t@aist.go.jp}
		\and
		\authorblockN{Shuuji Kajita$^{1}$}
		\authorblockA{s.kajita@aist.go.jp}
		\thanks{$^{1}$ R. Cisneros, T. Sakaguchi and S. Kajita are members of the
						Humanoid Research Group of the National Institute of Advanced Industrial Science
						and Technology (AIST), 305-8568 Tsukuba, Japan}}
  
	\maketitle

	\thispagestyle{empty}
	\pagestyle{empty}

	\begin{abstract}
		This paper presents the method used by our team, AIST-NEDO, at the DARPA Robotics Challenge (DRC) to deal with
		the requested manipulation tasks by means of a task-level teleoperation, by considering a degraded communication
		between the user and the robot and that the environment was not known in advance.
		The method basically consists on the use of 3D models of objects (from now on referred as ``markers'') which,
		once aligned with the actual attitude of the real objects that they represent, provide a reference frame in
		which the motion can be described, in order to successfully realize a manipulation task in a non-structured
		environment.
		These markers can represent the object being manipulated, some reference object in the environment and the
		hands of the robot.
		This method is illustrated by means of describing three representative tasks (which where requested during the DRC)
		and presenting the corresponding results obtained during the competition.
	\end{abstract}
	
	\section{Introduction and Motivation}
	
		Disaster response is attracting attention from the robotics research community, and even more since the
		Fukushima Daiichi nuclear power plant accident that followed the 2011 Great East Japan earthquake and tsunami.
		As a concrete materialization of this increasing interest, a challenge is proposed by the American Defense
		Advanced Research Projects Agency (DARPA) to use robots in disaster-hit facilities that were made too hazardous
		for direct human operator intervention.
		It is worth noticing that the challenge does not impose any constraint on the design of the robot, but since the
		environment (industrial ladders, doors, valves, cars) as well as the tools (levers, drills, hammers) were meant
		to comply with the human morphology, it is a natural option to develop the necessary means to make the humanoid
		robots capable of performing inspection and disaster recovering actions inside a non-structured environment
		\cite{Bouyarmane}.
		
		This environment can be considered to be ``kind of'' known in the sense that we know which actions
		(the type of tasks)	are required in advance and that we have a rough idea of its spatial distribution,
		maybe altered due to the disaster itself.
		However, it is assumed that there is no precise model of this environment, and that there are previously unknown
		obstacles, randomly placed on it.
		
		Despite of these conditions, the robot must be able to travel through the environment to achieve a proper stance
		with respect to the object(s) representing the target of the task, such that they be inside of the dextrous
		workspace of the hands of the robot.
		
		It is also mandatory to consider that within a desaster-hit facility it is not possible to rely on a stable,
		wide bandwidth communication system.
		
		
		This has to be done by means of a task-level teleoperation
		
	\section{Related Work}
	
	\section{Teleoperated Manipulation Method}
	
		\subsection{Manipulation Marker Alignment}
		
		\subsection{Manipulation Marker Based Motion Description}
		
		\subsection{Dealing with Uncertainties}
		
	\section{Examples}
		
		\subsection{Opening a Door}
		
		\subsection{Pulling and Inserting a Plug}
		
		\subsection{Opening a Box and Pressing a Button}
		
	\section{Results}
	
	\section{Conclusions}
	
	\bibliographystyle{unsrt}
	\bibliography{ManipulationDARPA}
	
\end{document}