\section{Introduction and Motivation}
	\label{sec:introduction}

	Disaster response is attracting attention from the robotics research community, and even more since the
	Fukushima Daiichi nuclear power plant accident that followed the 2011 Great East Japan earthquake and tsunami.
	As a concrete materialization of this increasing interest, a challenge is proposed by the American Defense
	Advanced Research Projects Agency (DARPA) to use robots in disaster-hit facilities that were made too hazardous
	for direct human operator intervention.
	It is worth noticing that the challenge does not impose any constraint on the design of the robot, but since the
	environment (industrial ladders, doors, valves, cars) as well as the tools (levers, drills, hammers) were meant
	to comply with the human morphology, it is a natural option to develop the necessary means to make the humanoid
	robots capable of performing inspection and disaster recovering actions inside a non-structured environment
	\cite{Bouyarmane}.
	
	This environment can be considered to be ``kind of'' known in the sense that we know which actions
	(the type of tasks)	are required in advance and that we have a rough idea of its spatial distribution,
	maybe altered due to the disaster itself.
	However, it is assumed that there is no precise model of this environment, and that there are previously unknown
	obstacles, randomly placed on it.
	
	Despite of these conditions, the robot must be able to travel through the environment to achieve a proper stance
	with respect to the object(s) representing the target of the task, such that they be inside of the dextrous
	workspace of the hands of the robot.
	
	It is also mandatory to consider that within a desaster-hit facility it is not possible to rely on a stable,
	wide bandwidth communication system.
	
	% Still working on this
	This has to be done by means of a task-level teleoperation