\section{Introduction and Motivation}
	\label{sec:introduction}

	Disaster response is attracting attention from the robotics research community, and even more since the
	Fukushima Daiichi nuclear power plant accident that followed the 2011 Great East Japan earthquake and tsunami.
	As a concrete materialization of this increasing interest, a challenge is proposed by the American Defense
	Advanced Research Projects Agency (DARPA) to use robots in disaster-hit facilities that were made too hazardous
	for direct human operator intervention.
	It is worth noticing that the challenge does not impose any constraint on the design of the robot, but since the
	environment (industrial ladders, doors, valves, cars) as well as the tools (levers, drills, hammers) were meant
	to comply with the human morphology, it is a natural option to develop the necessary means to make the humanoid
	robots capable of performing inspection and disaster recovering actions inside a non-structured environment
	\cite{Bouyarmane}.
	
	This environment can be considered to be ``kind of'' known in the sense that we know which actions
	are required in advance and that we have a rough idea of its spatial distribution,
	maybe altered due to the disaster itself.
	Given these conditions, only very limited assumptions about the structure of the environment can
	be made beforehand, in contrast to structured scenarios where semantic knowledge of their structure
	can be leveraged for highly autonomous robots operating in them~\cite{Kohlbrecher}.
	Furthermore, it is also mandatory to consider that within a disaster-hit facility it is not possible
	to rely on a stable, wide bandwidth wireless communication system with the robot.
	The signal may be degraded and blackouts may occur frequently.
	
	Then, it is not feasible to consider a purely teleoperated robot.
	First, because of the high dimensionality of its control system, and second because the capabilities of
	the robot and the operator should include near real-time feedback without disruptions in the communications
	as well as transmission of large amounts of data to the operator.
	On the other hand, a fully autonomous robot navigating and interacting in an unconstrained environment
	should include extensive databases of information about possible objects of interest to be found,
	highly efficient grasping algorithms and the ability to react to unforeseen situations,
	which are still unsolved problems~\cite{Romay}.
	A feasible alternative is the development of supervised semi-autonomous high Degrees-Of-Freedom (DOF)
	robotic systems; that is, task-level teleoperated systems in which the operator cognitive burden is
	minimized by lowering the control space dimensionality~\cite{Katyal}, such that these operators function
	as supervisors setting high level goals, assisting the robot with complex perception tasks, directly
	changing robot parameters to improve its performance and making decisions when facing unexpected
	situations~\cite{Kohlbrecher}.